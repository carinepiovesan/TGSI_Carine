\section{Acessibilidade Web}
\label{sec:acessibilidadeweb}


Nos \'ultimos anos, o desenvolvimento Web teve um grande crescimento visto que
a intera\c{c}\~ao com a Web vem se tornando comum no dia a dia das pessoas, seja para
trabalho, estudo ou entretenimento. Ao fazer um retrospecto da Web, podemos
visualizar um grande avan\c{c}o, pois passamos de p\'aginas est\'aticas para o
processamento de p\'aginas mais interativas e din\^amicas, permitindo agregar
facilidades e uma grande gama de recursos para disponibiliza\c{c}\~ao do
conte\'udo~\cite{Jazayeri}. Assim, tem-se criado diversos m\'etodos de suporte
para os desenvolvimentos de aplica\c{c}\~oes Web acess\'iveis~\cite{Freire}.

Apesar das t\'ecnicas de desenvolvimento de mais acess\'iveis, ainda encontramos
dificuldades na implementa\c{c}\~ao da cultura de acessibilidade, pois os
desenvolvedores ainda n\~ao se conscientizaram sobre a import\^ancia do tema Web
Acess\'ivel. A implanta\c{c}\~ao de acessibilidade demanda conhecimentos sobre
a vantagem e desvantagem de cada m\'etodo a ser implementado, exigindo muito estudo~\cite{Freire}.    
 
Segundo a W3C~\cite{WAI}, para que um site possa ser desenvolvido de forma
acess\'ivel, \'e necess\'ario que v\'arios componentes estejam interligados, muitos recursos de
acessibilidade podem ser implementados facilmente, desde que no in\'icio do
desenvolvimento do projeto sejam identificados os problemas que impedem a
acessibilidade web. Componentes estes que abrangem conte\'udo, c\'odigos de
marca\c{c}\~ao, ferramentas de cria\c{c}\~ao, desenvolvedores, tecnologias
espec\'ificas para cada defici\^encia quando necess\'ario, entre outros. A W3C, 
determina que acessibilidade web permite que diferentes usu\'arios tenham eles
qualquer tipo de defici\^encia tais como auditiva, cognitiva, visual,
f\'isica, de fala, entre outras, possam utilizar a Web e o conte\'udo nela
disponibilizada, ou seja, que possam perceber, compreender, interagir enavegar
na Web independente de suas habilidades e defici\^encias~\cite{WAI/W3C}.

O governo Brasileiro tamb\'em mostra muito interesse quando falamos em
acessibilidade web, podemos visualizar tal interesse quando no decreto N\textsuperscript{\underline{o}} 5.296
de 2 de dezembro de 2004. O governo determina o regulamento das Leis
N\textsuperscript{\underline{o}} 10.048, de 8 de novembro de 2000, e 10.098, de
19 de dezembro de 2000. Segundo o governo brasileiro acessibilidade se refere
\`a utiliza\c{c}\~ao total de todos os recursos fornecidos em qualquer \'area,
excluindo poss\'iveis barreiras que venham a limitar o acesso a qualquer
pessoa~\cite{AcessibilidadeBrasil}. Apesar da exist\^encia de diversas
diretrizes de acessibilidade e leis federais tais como a Lei
N\textsuperscript{\underline{o}} 10.048 que define acessibilidade ao fato de
estar relacionada em fornecer condi\c{c}\~ao para utiliza\c{c}\~ao, com seguran\c{c}a e
autonomia, total ou assistida, dos espa\c{c}os, mobili\'arios e equipamentos
urbanos, das edifica\c{c}\~oes, dos servi\c{c}os de transporte e dos dispositivos,
sistemas e meios de comunica\c{c}\~ao e informa\c{c}\~ao, por pessoa com defici\^encia ou
com mobilidade reduzida, muitos portais ainda possuem grandes limita\c{c}\~oes e
barreiras a algum determinado grupo de
usu\'arios~\cite{Freire,AcessibilidadeBrasil}.

A \emph{Lei de Acesso \`a Informa\c{c}\~ao} N\textsuperscript{\underline{o}} 12.527/2011 regulamenta o
direito constitucional de acesso \`as informa\c{c}\~oes p\'ublicas. A norma entrou em vigor em
16 de maio de 2012 criando mecanismos que possibilitam, a qualquer pessoa, 
f\'isica ou jur\'idica, sem necessidade de apresentar motivo, o recebimento de
informa\c{c}\~oes p\'ublicas dos \'org\~aos e entidades. A lei vale para os
tr\^es Poderes da Uni\~ao, Estados, Distrito Federal e Munic\'ipios, inclusive aos
Tribunais de Conta e Minist\'erio P\'ublico. Entidades privadas sem fins
lucrativos tamb\'em s\~ao obrigadas a dar publicidade a informa\c{c}\~oes referentes
ao recebimento e \`a destina\c{c}\~ao dos recursos p\'ublicos por elas
recebidos~\cite{LAI}.


\subsection{Padr\~oes de Acessibilidade Web}
\label{subsec:PadroesdeAcessibilidadeWeb}
 
As diretrizes de Acessibilidade para Conte\'udo Web (WCAG) 2.0, sucede a WCAG
1.0 que foi criado em 1999, abrangem uma grande variedade de recomenda\c{c}\~oes
que faz com que a web e seu conte\'udo sejam mais acess\'iveis \`as pessoas que
tenham algum tipo de defici\^encia, neste documento podemos encontrar afirma\c{c}\~oes que
podem ser testadas sem algum tipo de tecnologia espec\'ifica~\cite{Caldwell}. 

A WCAG 2.0 aborda quatro n\'iveis importantes para ajudar desenvolvedores e
organiza\c{c}\~oes a proporcionar ao seu p\'ublico alvo um conte\'udo mais
acess\'ivel \`a todos os tipos de usu\'arios, s\~ao estes os n\'iveis abordados
na vers\~ao WCAG 2.0:

\begin{itemize}
  \item \textbf{Princ\'ipios:} fornecem a base para um conte\'udo Web
  acess\'ivel na qual se enquadram as caracter\'isticas percept\'ivel, oper\'avel,
  compreens\'ivel e robusto;
  \item \textbf{Diretrizes:} fornecem informa\c{c}\~oes de como deixar o
  conte\'udo web mais acess\'ivel a usu\'arios com diferentes defici\^encias.
  \item \textbf{Crit\'erios de Sucesso:} s\~ao atribu\'idos a cada diretriz
  tr\^es n\'iveis de conformidade ou prioridade A, AA e AAA utilizados para
  satisfazer diferentes situa\c{c}\~oes encontradas durante o processo de
  desenvolvimento. Prioridade A: determina pontos nas quais os desenvolvedores
  Web DEVEM satisfazer inteiramente, Prioridade AA: Pontos nas quais os
  desenvolvedores DEVERIAM satisfazer, Prioridade AAA: Pontos nas quais os
  desenvolvedores PODEM satisfazer;
  \item \textbf{T\'ecnicas de tipo Necess\'aria e de tipo Sugerida:} s\~ao
  t\'ecnicas de testes necess\'arios para cumprir os crit\'erios de sucesso ao
  aplicar as recomenda\c{c}\~oes estabelecidas pela WCAG 2.0 e as t\'ecnicas de tipo
  sugerido s\~ao aquelas nas quais poderiam ser aplicados t\'ecnicas de testes
  \`as recomenda\c{c}\~oes, por\'em estas n\~ao s\~ao abrangidas pelos crit\'erios
  de sucesso test\'aveis~\cite{Caldwell}.
\end{itemize} 

O Modelo de Acessibilidade em Governo Eletr\^onico (E-MAG) \'e o modelo de
diretriz de acessibilidade brasileiro e  tem como compromisso, ser o norteador de
acessibilidade Web nos portais federais brasileiros, garantido o acesso as
informa\c{c}\~oes neles disponibilizados por toda a
popula\c{c}\~ao~\cite{eMAG}.O E-MAG possui recomenda\c{c}\~oes de acessibilidade
conforme as necessidades brasileiras e em conformidade com os padr\~oes de
diretrizes internacionais (WCAG),o documento \'e dividido em seis diretrizes que
seguem as recomenda\c{c}\~oes de acessibilidade determinadas pela WCAG 2.0,
s\~ao elas:

\begin{itemize}
  \item \textbf{Marca\c{c}\~ao:} \'e constitu\'ida por recomenda\c{c}\~oes que
  se referem ao cumprimento e respeito dos padr\~oes Web, envolvendo a pr\'atica da organiza\c{c}\~ao do
  c\'odigo e sua sem\^antica. 
  \item \textbf{Comportamento (Document Object Model - DOM):} constitu\'ida por
  recomenda\c{c}\~oes quanto aos comandos e meios de acesso as p\'aginas por
  meio do teclado.  
  \item \textbf{Conte\'udo/Informa\c{c}\~ao:} refere-se \`a recomenda\c{c}\~oes
  que dizem respeito ao conte\'udo disponibilizado pelo site, tais como idioma principal, t\'itulos
  descritivos, descri\c{c}\~ao de imagens garantia de entendimento do conte\'udo
  por todos os tipos de usu\'arios.  
  \item \textbf{Multim\'idia:} refere-se \`a recomenda\c{c}\~oes que dizem
  respeito aos v\'arios meios de transmitir as informa\c{c}\~oes e conte\'udos em formatos
  multim\'idia, sejam elas por v\'ideos, \'audios, anima\c{c}\~oes e
  audiodescri\c{c}\~oes.
  \item \textbf{Formul\'ario (E-MAG, 2014):} refere-se \`a recomenda\c{c}\~oes
  que dizem respeito aos formul\'arios, tais como reconhecimentos de erros dos dados de
  entrada e confirma\c{c}\~oes de envio dos dados, estrat\'egias de
  seguran\c{c}a para substitui\c{c}\~ao do CAPTCHA~\cite{eMAG}.  
\end{itemize}
 
 Atualmente, encontramos a especifica\c{c}\~ao WAI-ARIA
 \emph{(Accessible Rich Internet Applications Suite)}, na qual \'e uma
 recomenda\c{c}\~ao que trata de acessibilidade em interfaces ricas. A ARIA ajuda em conte\'udos 
 din\^amicos e controles avan\c{c}ados de interface, focando principalmente
 em usu\'arios que dependam de leitores de tela e usu\'arios que n\~ao possam
 utilizar o mouse~\cite{web}, \'e importante salientar que neste trabalho n\~ao
 esta sendo considerada esta especifica\c{c}\~ao. 


%Sommerville~\etal~\cite{Barbosa} e Wessberg~\cite{wessberg2000real}