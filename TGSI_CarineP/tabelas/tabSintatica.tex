\begin{table}[!h] 
 \tiny               
\centering
\caption{Regras Sint�ticas}
\label{tabSintatica}
\begin{tabular}{|l|l|}
\hline
\multicolumn{1}{|c|}{\multirow{2}{*}{ID}} & \multicolumn{1}{c|}{\multirow{2}{*}{Recomenda��o}}                                                                                 \\
\multicolumn{1}{|c|}{}                    & \multicolumn{1}{c|}{}                                                                                                              \\ \hline
1.1                                       & Respeitar os Padr�es Web                                                                                                           \\ \hline
1.3                                       & Utilizar corretamente os n�veis de cabe�alho                                                                                       \\ \hline
1.7                                       & Separar Links adjacentes                                                                                                           \\ \hline
1.9                                       & N�o abrir novas inst�nias sem a solicita��o do usu�rio                                                                             \\ \hline
2.2                                       & Garantir que os objetos program�veis sejam acess�veis                                                                              \\ \hline
2.3                                       & N�o criar p�ginas com atualiza��o autom�tica peri�dica                                                                             \\ \hline
2.4                                       & N�o criar redirecionamento autom�tico de p�ginas                                                                                   \\ \hline
2.5                                       & Fornecer alternativa para modificar limite de tempo                                                                                \\ \hline
2.7                                       & \begin{tabular}[c]{@{}l@{}}Assegurar o controle do usu�rio sobre \\ as altera��es temporais do conte�do\end{tabular}               \\ \hline
3.1                                       & Identificar o idioma principal da p�gina                                                                                           \\ \hline
3.2                                       & Informar mudan�as de idioma no conte�do                                                                                            \\ \hline
3.8                                       & Disponibilizar documentos em formatos acess�veis                                                                                   \\ \hline
3.9                                       & \begin{tabular}[c]{@{}l@{}}Em tabelas,  utilizar t�tulos e resumos de\\ forma apropriada\end{tabular}                              \\ \hline
4.1                                       & \begin{tabular}[c]{@{}l@{}}Oferecer contraste m�nimo entre plano\\  de fundo e primeio plano\end{tabular}                          \\ \hline
4.2                                       & \begin{tabular}[c]{@{}l@{}}N�o utilizar apenas cor ou outras caracter�sticas\\  sensoriais para diferenciar elementos\end{tabular} \\ \hline
4.4                                       & \begin{tabular}[c]{@{}l@{}}Possibilitar que o elemento em foco \\ seja visualmente evidente\end{tabular}                           \\ \hline
5.4                                       & Fornecer controle de �udio para som                                                                                                \\ \hline
5.5                                       & Fornecer controle de anima��o                                                                                                      \\ \hline
6.1                                       & \begin{tabular}[c]{@{}l@{}}Fornecer alternativa em texto para \\ os bot�es de imagem de formul�rios\end{tabular}                   \\ \hline
6.2                                       & Associar etiquetas aos seus campos                                                                                                 \\ \hline
6.3                                       & Estabelecer uma ordem l�gica de navega��o                                                                                          \\ \hline
6.4                                       & N�o provocar automaticamente altera��o no contexto                                                                                 \\ \hline
6.5                                       & Fornecer instru��es para entrada de dados                                                                                          \\ \hline
6.6                                       & \begin{tabular}[c]{@{}l@{}}Identificar e descrever erros de entrada de \\ dados e confirmar o envio de informa��es\end{tabular}    \\ \hline
1.44                                      & Agrupar campos de formul�rios                                                                                                      \\ \hline
\end{tabular}
\end{table}